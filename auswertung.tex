\section{Auswertung}
\label{sec:Auswertung}

\subsection{Fehlerrechnung}
\label{sec:Fehlerrechnung}
Für die Fehlerrechnung werden folgende Formeln aus der Vorlesung verwendet.
für den Mittelwert gilt
\begin{equation}
    \overline{x}=\frac{1}{N}\sum_{i=1}^N x_i ß\; \;\text{mit der Anzahl N und den Messwerten x} 
    \label{eqn:Mittelwert}
\end{equation}
Der Fehler für den Mittelwert lässt sich gemäß
\begin{equation}
    \increment \overline{x}=\frac{1}{\sqrt{N}}\sqrt{\frac{1}{N-1}\sum_{i=1}^N(x_i-\overline{x})^2}
    \label{eqn:FehlerMittelwert}
\end{equation}
berechnen.
Wenn im weiteren Verlauf der Berechnung mit der fehlerhaften Größe gerechnet wird, kann der Fehler der folgenden Größe
mittels Gaußscher Fehlerfortpflanzung berechnet werden. Die Formel hierfür ist
\begin{equation}
    \increment f= \sqrt{\sum_{i=1}^N\left(\frac{\partial f}{\partial x_i}\right)^2\cdot(\increment x_i)^2}.
    \label{eqn:GaussMittelwert}
\end{equation}

\subsection{Bestimmung des Brechungsindex}
Es werden aus den Messungen zur parallelen und senkrechten Polarisation der Brechungsindex bestimmt. Zur Messung wurde ein Dunkelstrom 
$I_{\symup{D}} = 3\,\unit{\nano\ampere}$ gemessen. Dieser ist gegenüber der Messdaten vernachlässigbar klein.

\subsubsection{Senkrechte Polarisation}
Zuerst muss die gemessene Intensität umgerechnet werden. Es gilt
\begin{equation}
  \frac{I_{\symup{r}}}{I_{\symup{e}}} \thicksim \frac{E_{\symup{r}}^2}{E_{\symup{e}}^2} = E^2 \. .
    \label{eqn:energie}
\end{equation}
Dabei ist $I_e = 200 \, \unit{\micro\ampere}$. Die Messdaten zum Einfallswinkel, zur Intensität und zur Wurzel des Intensitätsverhältnisses 
$\frac{I_{\symup{r}}}{I_{\symup{e}}}$ sind in \autoref{tab:Pol0} dargestellt.
Des Weiteren wird mit \autoref{eqn: n1} und obiger Beziehung der Brechungsindex jeder einzelnen Messung berechnet und ebenfalls in \autoref{tab:Pol0} abgebildet.

\begin{table}
    \centering
    \caption{Messwerte der senkrechten Polarisation sowie der Brechungsindex n.}
    \begin{tabular}{c c c c}
        \toprule
        $\alpha \mathrm{/} \unit{\degree}$  & $I_{\symup{r}}\mathrm{/}\unit{\micro\ampere}$ & $\frac{I_{\symup{r}}}{I_{\symup{e}}}$ & n\\
        \midrule
        6&67&0.5788&3.6098 \\
        8&70&0.5916&1.1403 \\
       10&66&0.5745&3.1518 \\
       12&72&0.6000&3.4178 \\
       14&67&0.5788&1.1153 \\
       16&70&0.5916&3.7433 \\
       18&73&0.6042&2.7793 \\
       20&76&0.6164&1.9471 \\
       22&78&0.6245&4.3261 \\
       24&72&0.6000&1.9233 \\
       26&80&0.6325&2.9728 \\
       28&74&0.6083&3.9614 \\
       30&80&0.6325&1.2023 \\
       32&80&0.6325&3.7460 \\
       34&82&0.6403&3.9058 \\
       36&84&0.6481&1.1588 \\
       38&85&0.6519&4.5423 \\
       40&90&0.6708&3.4662 \\
       42&92&0.6782&2.2786 \\
       44&93&0.6819&5.2867 \\
       \bottomrule
    \end{tabular}
    \quad
    \begin{tabular}{c c c c}
        \toprule
        $\alpha \mathrm{/} \unit{\degree}$  & $I_{\symup{r}}\mathrm{/}\unit{\micro\ampere}$ & $\frac{I_{\symup{r}}}{I_{\symup{e}}}$ & n\\
        \midrule
       46&97&0.6964&2.5779 \\
       48&99&0.7036&3.7581 \\
      50&100&0.7071&5.6304 \\
      52&100&0.7071&1.3696 \\
      54&100&0.7071&4.8658 \\
      56&100&0.7071&5.0002 \\
      58&100&0.7071&1.2117 \\
      60&100&0.7071&5.5594 \\
      62&100&0.7071&3.9945 \\
      64&110&0.7416&2.7970 \\
      66&120&0.7746&7.8703 \\
      68&120&0.7746&3.5797 \\
      70&120&0.7746&5.0458 \\
      72&130&0.8062&9.0196 \\
      74&140&0.8367&2.1677 \\
      76&140&0.8367&9.2864 \\
      78&140&0.8367&9.6592 \\
      80&140&0.8367&1.5901 \\
     82&160&0.8944&17.0442 \\
     84&160&0.8944&12.2245 \\
        \bottomrule
    \end{tabular}
    \label{tab:Pol0}
\end{table}

Der gemittelte Brechungsindex ergibt sich zu $\symup{n} = 4.3482$.

\subsubsection{parallele Polarisation}
Erneut wird die Intensität mittels \autoref{eqn:energie} umgerechnet. Dabei ist $I_e = 140 \, \unit{\micro\ampere}$. Die Messwerte sind in 
\autoref{tab:Pol90} dargestellt.
\begin{table}
    \centering
    \caption{Messwerte der parallelen Polarisation sowie der Brechungsindex n.}
    \begin{tabular}{c c c c}
        \toprule
        $\alpha \mathrm{/} \unit{\degree}$  & $I_{\symup{r}}\mathrm{/}\unit{\micro\ampere}$ & $\frac{I_{\symup{r}}}{I_{\symup{e}}}$ & n\\
        \midrule
        6&54.0&0.6211&4.4499 \\
        8&54.0&0.6211&29.3909 \\
        10&54.0&0.6211&5.0803 \\
        12&53.0&0.6153&4.9574 \\
       14&52.0&0.6094&30.1278 \\
        16&50.0&0.5976&4.1396 \\
        18&50.0&0.5976&5.9831 \\
        20&50.0&0.5976&9.7023 \\
        22&49.0&0.5916&3.8974 \\
        24&48.0&0.5855&8.9898 \\
        26&45.0&0.5669&5.5596 \\
        28&44.0&0.5606&3.6834 \\
       30&44.0&0.5606&23.0123 \\
        32&43.0&0.5542&4.1554 \\
        34&41.0&0.5412&3.9350 \\
       36&37.0&0.5141&24.3370 \\
        38&39.0&0.5278&3.3791 \\
        40&38.0&0.5210&4.7218 \\
        42&36.0&0.5071&7.6073 \\
        44&34.0&0.4928&2.9437 \\
       \bottomrule
    \end{tabular}
    \quad
    \begin{tabular}{c c c c}
        \toprule
        $\alpha \mathrm{/} \unit{\degree}$  & $I_{\symup{r}}\mathrm{/}\unit{\micro\ampere}$ & $\frac{I_{\symup{r}}}{I_{\symup{e}}}$ & n\\
        \midrule
        46&32.0&0.4781&6.5104 \\
        48&30.0&0.4629&4.2061 \\
        50&29.0&0.4551&2.7589 \\
       52&26.0&0.4309&15.4058 \\
        54&22.0&0.3964&2.7481 \\
        56&22.0&0.3964&2.6743 \\
       58&20.0&0.3780&18.5681 \\
        60&17.0&0.3485&2.1569 \\
        62&14.0&0.3162&2.7803 \\
        64&10.0&0.2673&4.3344 \\
         66&8.0&0.2390&1.6287 \\
         68&6.7&0.2188&3.4441 \\
         70&4.6&0.1813&2.1495 \\
         72&2.5&0.1336&1.3309 \\
         74&1.4&0.1000&7.0543 \\
         76&1.0&0.0845&1.3069 \\
         78&2.0&0.1195&1.3886 \\
        80&5.2&0.1927&13.3533 \\
        82&12.0&0.2928&1.9046 \\
        84&22.0&0.3964&3.3430 \\
        86&46.0&0.5732&9.5785 \\
        \bottomrule
    \end{tabular}
    \label{tab:Pol90}
\end{table}
Der Brechungsindex berechnet sich dabei über \autoref{eqn:n90} und wird ebenfalls in der Tabelle dargestellt. 
Der Mittelwert der Brechungsindizes ergibt sich so zu $\bar{n}= 7.4670$.

Außerdem lässt sich mittels der Messung zur parallelen Polarisation der Brewsterwinkel bestimmen. Dieser befindet sich beim Minimum der 
reflektierten Intensität und liegt somit bei $\alpha_{\symup{p}} = 76°$. Mittels \autoref{eqn:brewster} berechnet sich der Brechungsindex somit zu  
$\symup{n} = 4.0107$.

\subsection{Plot der Messwerte und Vergleich mit der Theorie}
%Um die Theoriekurven darstellen zu können wird erneut der Mittelwert der gemittelten Brechungsindizes bestimmt. Dabei wird jedoch der durch die Messung
%der parallelen Polarisation bestimmte Brechungsindex $\bar{n}= 7.4670$ aus der Mittelung ausgenommen, da die Abweichung zu den anderen beiden gemittelten
%Brechungsindizes zu groß ist.  Damit ergibt sich ein Brechungsindex von
%$\bar{\symup{n}} = 4.1795$.

In \autoref{fig:plot0} wird nun $\sqrt{I/I_0}$ gegen den Einfallswinkel $\alpha$ aufgetragen. Die Theoriekurve berechnet sich durch \autoref{eqn:fresnelS2}. 
\begin{figure}
    \centering
    \includegraphics[height = 10cm]{build/plot0Pol.pdf}
    \caption{Vergleich der Messwerte mit der Theorie der senkrechten Polarisation.}
    \label{fig:plot0}
\end{figure}

In \autoref{fig:plot90} wird $\sqrt{I/I_0}$ der parallelen Polarisation gegen den Einfallswinkel $\alpha$ aufgetragen. Die Theoriekurve berechnet sich diesmal 
durch \autoref{eqn:fresnelP2}.
\begin{figure}
    \centering
    \includegraphics[height = 10cm]{build/plot90Pol.pdf}
    \caption{Vergleich der Messwerte mit der Theorie der parallelen Polarisation.}
    \label{fig:plot90}
\end{figure}

Der Brewsterwinkel kann nicht nru aus den Messwerten der parallelen Polarisation abgelesen werden, sondern auch über die bestimmten Brechungsindizes berechnet werden.
Die Formel
\begin{equation*}
    \alpha_{\text{B}}= \tan^{-1}\left(\frac{n_2}{n_1}\right)
\end{equation*}
beschreibt den Zusammenhang des Brewsterwinkels und den Brechungsindizes der Materialien an der Grenzfläche und kann aus dem Brechungsgesetz hergeleitet werden. Für $n_1$ wird der Brechungsindex von Luft verwendet, welcher Näherungsweise einen Wert von $n_1 = 1$ besitzt.
So ergeben sich folgende Werten
\begin{align*}
    \alpha_{\text{B,senkrecht}}&=77.05\, °\. ,\\
    \alpha_{\text{B,parallel}}&=82.37\, °\. .
    \end{align*}