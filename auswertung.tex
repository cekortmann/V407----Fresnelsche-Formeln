\section{Auswertung}
\label{sec:Auswertung}

\subsection{Fehlerrechnung}
\label{sec:Fehlerrechnung}
Für die Fehlerrechnung werden folgende Formeln aus der Vorlesung verwendet.
für den Mittelwert gilt
\begin{equation}
    \overline{x}=\frac{1}{N}\sum_{i=1}^N x_i ß\; \;\text{mit der Anzahl N und den Messwerten x} 
    \label{eqn:Mittelwert}
\end{equation}
Der Fehler für den Mittelwert lässt sich gemäß
\begin{equation}
    \increment \overline{x}=\frac{1}{\sqrt{N}}\sqrt{\frac{1}{N-1}\sum_{i=1}^N(x_i-\overline{x})^2}
    \label{eqn:FehlerMittelwert}
\end{equation}
berechnen.
Wenn im weiteren Verlauf der Berechnung mit der fehlerhaften Größe gerechnet wird, kann der Fehler der folgenden Größe
mittels Gaußscher Fehlerfortpflanzung berechnet werden. Die Formel hierfür ist
\begin{equation}
    \increment f= \sqrt{\sum_{i=1}^N\left(\frac{\partial f}{\partial x_i}\right)^2\cdot(\increment x_i)^2}.
    \label{eqn:GaussMittelwert}
\end{equation}

\subsection{Bestimmung des Brechungsindex}
Es werden aus den Messungen zur parallelen und senkrechten Polarisation der Brechungsindex bestimmt. Zur Messung wurde ein Dunkelstrom 
$I_{\symup{D}} = 3\,\unit{\nano\ampere}$ gemessen. Dieser ist gegenüber der Messdaten vernachlässigbar klein.

\subsubsection{Senkrechte Polarisation}
Zuerst muss die gemessene Intensität umgerechnet werden. Es gilt
\begin{equation}
  \frac{I_{\symup{r}}}{I_{\symup{e}}} \thicksim \frac{E_{\symup{r}}^2}{E_{\symup{e}}^2} = E^2 \. .
    \label{eqn:energie}
\end{equation}
Dabei ist $I_e = 200 \, \unit{\micro\ampere}$. Die Messdaten zum Einfallswinkel, zur Intensität und zur Wurzel des Intensitätsverhältnisses 
$\frac{I_{\symup{r}}}{I_{\symup{e}}}$ sind in \autoref{tab:Pol0} dargestellt.
Desweiteren wird mit \autoref{eqn: n1} und obiger Beziehung der Brechungsindex jeder einzelnen Messung berechnet und ebenfalls in \autoref{tab:Pol0} abgebildet.

\begin{table}
    \centering
    \caption{Messwerte der senkrechten Polarisation sowie der Brechungsindex n.}
    \begin{tabular}{c c c c}
        \toprule
        $\alpha \mathrm{/} \unit{\degree}$  & $I_{\symup{r}}\mathrm{/}\unit{\micro\ampere}$ & $\frac{I_{\symup{r}}}{I_{\symup{e}}}$ & n\\
        \midrule
        6&67&0.5788&3.6098 \\
        8&70&0.5916&1.1403 \\
       10&66&0.5745&3.1518 \\
       12&72&0.6000&3.4178 \\
       14&67&0.5788&1.1153 \\
       16&70&0.5916&3.7433 \\
       18&73&0.6042&2.7793 \\
       20&76&0.6164&1.9471 \\
       22&78&0.6245&4.3261 \\
       24&72&0.6000&1.9233 \\
       26&80&0.6325&2.9728 \\
       28&74&0.6083&3.9614 \\
       30&80&0.6325&1.2023 \\
       32&80&0.6325&3.7460 \\
       34&82&0.6403&3.9058 \\
       36&84&0.6481&1.1588 \\
       38&85&0.6519&4.5423 \\
       40&90&0.6708&3.4662 \\
       42&92&0.6782&2.2786 \\
       44&93&0.6819&5.2867 \\
       \bottomrule
    \end{tabular}
    \quad
    \begin{tabular}{c c c c}
        \toprule
        $\alpha \mathrm{/} \unit{\degree}$  & $I_{\symup{r}}\mathrm{/}\unit{\micro\ampere}$ & $\frac{I_{\symup{r}}}{I_{\symup{e}}}$ & n\\
        \midrule
       46&97&0.6964&2.5779 \\
       48&99&0.7036&3.7581 \\
      50&100&0.7071&5.6304 \\
      52&100&0.7071&1.3696 \\
      54&100&0.7071&4.8658 \\
      56&100&0.7071&5.0002 \\
      58&100&0.7071&1.2117 \\
      60&100&0.7071&5.5594 \\
      62&100&0.7071&3.9945 \\
      64&110&0.7416&2.7970 \\
      66&120&0.7746&7.8703 \\
      68&120&0.7746&3.5797 \\
      70&120&0.7746&5.0458 \\
      72&130&0.8062&9.0196 \\
      74&140&0.8367&2.1677 \\
      76&140&0.8367&9.2864 \\
      78&140&0.8367&9.6592 \\
      80&140&0.8367&1.5901 \\
     82&160&0.8944&17.0442 \\
     84&160&0.8944&12.2245 \\
        \bottomrule
    \end{tabular}
    \label{tab:MesswerteRef}
\end{table}

Der gemittelte Brechungsindex ergibt sich zu $\symup{n} = 4.3482$.

\subsubsection{parallele Polarisation}
Erneut wird die Intensität mittels \autoref{eqn:energie} umgerechnet. Dabei ist $I_e = 140 \, \unit{\micro\ampere}$. 