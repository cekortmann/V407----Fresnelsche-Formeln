\section{Auswertung}
\label{sec:Auswertung}

\input{fehlerrechnung.tex}

\subsection{Bestimmung des Brechungsindex}
Es werden aus den Messungen zur parallelen und senkrechten Polarisation der Brechungsindex bestimmt. Zur Messung wurde ein Dunkelstrom 
$I_{\symup{D}} = 3\,\unit{\nano\ampere}$ gemessen. Dieser ist gegenüber der Messdaten vernachlässigbar klein.

\subsubsection{Senkrechte Polarisation}
Zuerst muss die gemessene Intensität umgerechnet werden. Es gilt
\begin{equation}
  \frac{I_{\symup{r}}}{I_{\symup{e}}} \thicksim \frac{E_{\symup{r}}^2}{E_{\symup{e}}^2} = E^2 \. .
    \label{eqn:energie}
\end{equation}
Dabei ist $I_e = 200 \, \unit{\micro\ampere}$. Die Messdaten zum Einfallswinkel, zur Intensität und zur Wurzel des Intensitätsverhältnisses 
$\frac{I_{\symup{r}}}{I_{\symup{e}}}$ sind in \autoref{tab:Pol0} dargestellt.
Des Weiteren wird mit \autoref{eqn: n1} und obiger Beziehung der Brechungsindex jeder einzelnen Messung berechnet und ebenfalls in \autoref{tab:Pol0} abgebildet.

\begin{table}
    \centering
    \caption{Messwerte der senkrechten Polarisation sowie der Brechungsindex n.}
    \begin{tabular}{c c c c}
        \toprule
        $\alpha \mathrm{/} \unit{\degree}$  & $I_{\symup{r}}\mathrm{/}\unit{\micro\ampere}$ & $\frac{I_{\symup{r}}}{I_{\symup{e}}}$ & n\\
        \midrule
        6&67&0.5788&3.6098 \\
        8&70&0.5916&1.1403 \\
       10&66&0.5745&3.1518 \\
       12&72&0.6000&3.4178 \\
       14&67&0.5788&1.1153 \\
       16&70&0.5916&3.7433 \\
       18&73&0.6042&2.7793 \\
       20&76&0.6164&1.9471 \\
       22&78&0.6245&4.3261 \\
       24&72&0.6000&1.9233 \\
       26&80&0.6325&2.9728 \\
       28&74&0.6083&3.9614 \\
       30&80&0.6325&1.2023 \\
       32&80&0.6325&3.7460 \\
       34&82&0.6403&3.9058 \\
       36&84&0.6481&1.1588 \\
       38&85&0.6519&4.5423 \\
       40&90&0.6708&3.4662 \\
       42&92&0.6782&2.2786 \\
       44&93&0.6819&5.2867 \\
       \bottomrule
    \end{tabular}
    \quad
    \begin{tabular}{c c c c}
        \toprule
        $\alpha \mathrm{/} \unit{\degree}$  & $I_{\symup{r}}\mathrm{/}\unit{\micro\ampere}$ & $\frac{I_{\symup{r}}}{I_{\symup{e}}}$ & n\\
        \midrule
       46&97&0.6964&2.5779 \\
       48&99&0.7036&3.7581 \\
      50&100&0.7071&5.6304 \\
      52&100&0.7071&1.3696 \\
      54&100&0.7071&4.8658 \\
      56&100&0.7071&5.0002 \\
      58&100&0.7071&1.2117 \\
      60&100&0.7071&5.5594 \\
      62&100&0.7071&3.9945 \\
      64&110&0.7416&2.7970 \\
      66&120&0.7746&7.8703 \\
      68&120&0.7746&3.5797 \\
      70&120&0.7746&5.0458 \\
      72&130&0.8062&9.0196 \\
      74&140&0.8367&2.1677 \\
      76&140&0.8367&9.2864 \\
      78&140&0.8367&9.6592 \\
      80&140&0.8367&1.5901 \\
     82&160&0.8944&17.0442 \\
     84&160&0.8944&12.2245 \\
        \bottomrule
    \end{tabular}
    \label{tab:Pol0}
\end{table}

Der gemittelte Brechungsindex ergibt sich zu $\symup{n} = 4.3482$.

\subsubsection{parallele Polarisation}
Erneut wird die Intensität mittels \autoref{eqn:energie} umgerechnet. Dabei ist $I_e = 140 \, \unit{\micro\ampere}$. Die Messwerte sind in 
\autoref{tab:Pol90} dargestellt.
\begin{table}
    \centering
    \caption{Messwerte der parallelen Polarisation sowie der Brechungsindex n.}
    \begin{tabular}{c c c c}
        \toprule
        $\alpha \mathrm{/} \unit{\degree}$  & $I_{\symup{r}}\mathrm{/}\unit{\micro\ampere}$ & $\frac{I_{\symup{r}}}{I_{\symup{e}}}$ & n\\
        \midrule
        6 & 54 & 0.5196 & 3.0502 \\
        8 & 54 & 0.5196 & 1.0912 \\
       10 & 54 & 0.5196 & 2.7094 \\
       12 & 53 & 0.5147 & 2.6885 \\
       14 & 52 & 0.5099 & 1.0765 \\
        16 & 50 & 0.5000 & 2.8874 \\
        18 & 50 & 0.5000 & 2.1185 \\
        20 & 50 & 0.5000 & 1.5272 \\
       22 & 49 & 0.4950 & 2.9601 \\
       24 & 48 & 0.4899 & 1.5346 \\
       26 & 45 & 0.4743 & 1.9682 \\
       28 & 44 & 0.4690 & 2.6771 \\
       30 & 44 & 0.4690 & 1.0763 \\
       32 & 43 & 0.4637 & 2.3425 \\
       34 & 41 & 0.4528 & 2.3141 \\
       36 & 37 & 0.4301 & 1.0425 \\
       38 & 39 & 0.4416 & 2.4834 \\
       40 & 38 & 0.4359 & 1.8540 \\
       42 & 36 & 0.4243 & 1.3487 \\
       44 & 34 & 0.4123 & 2.4028 \\
       \bottomrule
    \end{tabular}
    \quad
    \begin{tabular}{c c c c}
        \toprule
        $\alpha \mathrm{/} \unit{\degree}$  & $I_{\symup{r}}\mathrm{/}\unit{\micro\ampere}$ & $\frac{I_{\symup{r}}}{I_{\symup{e}}}$ & n\\
        \midrule
        46 & 32 & 0.4000 & 1.3528 \\
        48 & 30 & 0.3873 & 1.6405 \\
        50 & 29 & 0.3808 & 2.1677 \\
         52 & 26 & 0.3606 & 1.0458 \\
        54 & 22 & 0.3317 & 1.7443 \\
        56 & 22 & 0.3317 & 1.7782 \\
        58 & 20 & 0.3162 & 1.0190 \\
        60 & 17 & 0.2915 & 1.7629 \\
        62 & 14 & 0.2646 & 1.3739 \\
        64 & 10 & 0.2236 & 1.1081 \\
        66 & 8 & 0.2000 & 1.4997 \\
       68 & 6.7 & 0.1830 & 1.1011 \\
       70 & 4.6 & 0.1517 & 1.1568 \\
       72 & 2.5 & 0.1118 & 1.2371 \\
       74 & 1.4 & 0.0837 & 1.0059 \\
        76 & 1 & 0.0707 & 1.1057 \\
        78 & 2 & 0.1000 & 1.1676 \\
       80 & 5.2 & 0.1612 & 1.0056 \\
        82 & 12 & 0.2449 & 1.5969 \\
        84 & 22 & 0.3317 & 1.5406 \\
        86 & 46 & 0.4796 & 1.4293\\
        \bottomrule
    \end{tabular}
    \label{tab:Pol90}
\end{table}
Der Brechungsindex berechnet sich dabei über \autoref{eqn:n90} und wird ebenfalls in der Tabelle dargestellt. 
Der Mittelwert der Brechungsindizes ergibt sich so zu $\bar{n}= 1.75$.

Außerdem lässt sich mittels der Messung zur parallelen Polarisation der Brewsterwinkel bestimmen. Dieser befindet sich beim Minimum der 
reflektierten Intensität und liegt somit bei $\alpha_{\symup{p}} = 76°$. Mittels \autoref{eqn:brewster} berechnet sich der Brechungsindex somit zu  
$\symup{n} = 4.0107$.

\subsection{Plot der Messwerte und Vergleich mit der Theorie}
Um die Theoriekurven darstellen zu können wird erneut der Mittelwert aller drei gemittelten Brechungsindizes bestimmt. Damit ergibt sich 
$\bar{\symup{n}} = 4.1795$.

In \autoref{fig:plot0} wird nun $\sqrt{I/I_0}$ gegen den Einfallswinkel $\alpha$ aufgetragen. Die Theoriekurve berechnet sich durch \autoref{eqn:fresnelS2}. 
\begin{figure}
    \centering
    \includegraphics[height = 10cm]{build/plot0Pol.pdf}
    \caption{Vergleich der Messwerte mit der Theorie der senkrechten Polarisation.}
    \label{fig:plot0}
\end{figure}

In \autoref{fig:plot90} wird $\sqrt{I/I_0}$ der parallelen Polarisation gegen den Einfallswinkel $\alpha$ aufgetragen. Die Theoriekurve berechnet sich diesmal 
durch \autoref{eqn:fresnelP2}.
\begin{figure}
    \centering
    \includegraphics[height = 10cm]{build/plot90Pol.pdf}
    \caption{Vergleich der Messwerte mit der Theorie der parallelen Polarisation.}
    \label{fig:plot90}
\end{figure}