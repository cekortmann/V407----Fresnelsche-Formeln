\section{Durchführung}
\label{sec:Durchführung}
Vor Beginn der Messung muss die Messapparatur justiert werden. Hierzu wird zunächst der Detektor so eingestellt,
dass der Laser exakt in dessen Öffnung strahlt. 
Nun muss der Filter eingesetzt werden, und es wir der Polarisationswinkel gemessen, unter welchem sich ein Minimum ergibt.
Im Anschluss muss dieser auf $0\unit{\degree}$ gestellt.
Danach wird der Spiegel eingesetzt, sodass die reflektierende Fläche senkrecht zum Leser steht.
Jetzt wird die Neigung des Spiegels so angepasst, dass die Reflexion des Lasers in dessen Öffnung strahlt, damit auch der Spiegel die exakte 
Höhe und Neigung hat.
Das Goniometer wird so justiert, dass ein zuverlässiges Ablesen möglich ist. Hierbei bietet sich an, die $0\unit{\degree}$-Marke in eine Reihe mit Laser und dem Spiegel zu stellen.
Jetzt werden Detektor und Spiegel so gedreht, dass die Reflexion des Spiegels bei kleinst möglichem Winkel in den Detektor trifft. Dieser Winkel liegt bei den verwendeten 
Bauteilen bei rund $6\unit{\degree}$.
Die gemessene Stromstärke wird notiert und der Winkel des Spiegels um $2\unit{\degree}$ erhöht. Dieser Vorgang muss so lange wiederholt werden bis keine Messung mehr Möglich ist ($\alpha <90\unit{\degree})$.
Die Messung wird für einen Polarisationswinkel $90\unit{\degree}$ analog durchgeführt.