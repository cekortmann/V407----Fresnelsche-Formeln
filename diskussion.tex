\section{Diskussion}
\label{sec:Diskussion}

In den beiden graphischen Darstellungen \autoref{fig:plot0} und \autoref{fig:plot90} ist zu sehen, dass die Form der Messwerte jener der jeweiligen Theoriekurve entsprechen. Bei beiden graphischen Darstellungen ist zu sehen, dass die Kurven nicht genau in den Messwerten liegen.
Die Brechungsindizes, die über die beiden Polarisationsrichtung und den Brewsterwinkel bestimmt wurden, unterscheiden sich jedoch deutlicher von dem Theoriewert $n_{\text{theo}}=3.353$. Mit
experimentell bestimmten Werten von $\bar{n}=7.4670$, $\bar{n}= 4.0107$ und $\bar{n}= 4.1795$ liegen die Abweichungen zum Theoriewert bei $122.7\%$,  $19.6\%$ bzw. $24.7\%$. 
Der aus den Messwerten abgelesene Brewsterwinkel liegt bei $76\, °$, der Theoriewert, welcher über den Brechungsindex von Silizium berechnet wurde, liegt bei $73.39\, °$. Hieraus folgt eine Abweichung des abgelesenen Wertes von der Theorie von $3.6 \%$.
Darüber hinaus wurde der Brewsterwinkel auch über die unterschiedlichen Brechunsgindizes berechnet. Die von dem Brechungsindex abhängigen Winkel der Totalreflexion ergaben sich so zu $\alpha_{\text{B,senkrecht}}=77.05\, °$ und $\alpha_{\text{B,parallel}}=82.37\, °$.
Die Abweichungen ergeben sich so zu $5 \%$ bzw. $12.2\%$. Aufgrund der Abhängigkeit von den Brechungsindizes, zeichnet sich hier der Trend der Abweichungen ab.

Eine mögliche Ursache dafür kann neben der teilweise geringen
Genauigkeit beim Ablesen auch an möglichen Ungenauigkeiten beim Drehen der Vorrichtung sein. So konnte beim Rotieren des Spiegels und des Detektors auch ein geringes Mitdrehen des Goniometers nicht gänzlich ausgeschlossen werden.
Hinzukommen veränderliche Lichtverhältnisse, die einen Einfluss auf die Messung des Photoelements gehabt haben könnten. Diese Fehlerquelle war trotz eines vorheringen Bestimmens des Dunkelstroms nicht vollständig zu eliminieren.

Zusammenfassend kann also festgehalten werden, dass trotz der Bemühungen eine möglichst genaue Messreihe aufzunehmen, einige Fehler und Ungenauigkeiten nur minimiert, nicht aber ausgeschlossen werden konnten.