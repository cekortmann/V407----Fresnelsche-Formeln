\section{Diskussion}
\label{sec:Diskussion}

In den beiden graphischen Darstellungen \autoref{fig:plot0} und \autoref{fig:plot90} ist zusehen, dass die Messwerte relativ nah an der Theoriekurve liegen. 
Der Brechungsgindex der über die beiden Polarisationsrichtung und den Brewsterwinkel bestimmt wurde, unterscheiden sich jedoch deutlicher von dem Theoriewert $n_{\text{theo}}=3.353$. Mit
experimentell bestimmten Werten von $\bar{n}=7.4670$, $\bar{n}= 4.0107$ und $\bar{n}= 4.1795$ liegt die Abweichung zum Theoriewert bei $122.7\%$,  $19.6\%$ bzw. $24.7\%$. Eine mögliche Ursache dafür kann neben der teilweise geringen
Genauigkeit beim Ablesen auch an möglichen Ungenauigkeiten beim Drehen der Vorrichtung entstanden sein. So konnte beim Rotieren des Spiegels und des Detektors auch geringes ein Mitdrehen des Goniometers nicht gänzlich ausgeschlossen werden.
Hinzukommen veränderliche Lichtverhältnisse, die einen Einfluss auf die Messung des Photoelements gehabt haben könnten. Diese Fehlerquelle war trotz eines vorheringen Bestimmens des Dunkelstroms nicht vollständig zu eliminieren.

Zusammenfassend kann also festgehalten werden, dass trotz der Bemühungen eine möglichst genaue Messreihe aufzunehmen, einige Fehler und Ungenauigkeiten nur minimiert, nicht aber ausgeschlossen werden konnten.