\section{Diskussion}
\label{sec:Diskussion}

In den beiden graphischen Darstellungen \autoref{fig:plot0} und \autoref{fig:plot90} ist zu sehen, dass die Messwerte relativ nah an der Theoriekurve liegen. Gerade bei der
parallelen Polarisation beschreibt die Theoriekurve die Messwerte sehr gut.  Bei der senkrechten Polarisation gibt es jedoch einen größeren Abstand zur
Theoriekurve. Zudem schwanken die Messwerte etwas, stellen aber dennoch einen klaren Kurvenverlauf dar. Beides könnte an Messfehlern liegen.
Die Brechungsindizes, die über die beiden Polarisationsrichtung und den Brewsterwinkel bestimmt wurden, unterscheiden sich jedoch deutlicher von dem Theoriewert $n_{\text{theo}}=3.353$. Mit
experimentell bestimmten Werten von $\bar{n}=7.4670$, $\bar{n}= 4.0107$ und $\bar{n}= 4.1795$ liegen die Abweichungen zum Theoriewert bei $122.7\%$,  $19.6\%$ bzw. $24.7\%$. Eine mögliche Ursache dafür kann neben der teilweise geringen
Genauigkeit beim Ablesen auch an möglichen Ungenauigkeiten beim Drehen der Vorrichtung entstanden sein. So konnte beim Rotieren des Spiegels und des Detektors auch ein geringes Mitdrehen des Goniometers nicht gänzlich ausgeschlossen werden.
Hinzukommen veränderliche Lichtverhältnisse, die einen Einfluss auf die Messung des Photoelements gehabt haben könnten. Diese Fehlerquelle war trotz eines vorheringen Bestimmens des Dunkelstroms nicht vollständig zu eliminieren.

Zusammenfassend kann also festgehalten werden, dass trotz der Bemühungen eine möglichst genaue Messreihe aufzunehmen, einige Fehler und Ungenauigkeiten nur minimiert, nicht aber ausgeschlossen werden konnten.